\documentclass{article}

\usepackage[utf8]{inputenc}
\usepackage[T1]{fontenc}

\usepackage{tikz-er2}
\usetikzlibrary{positioning}
\usetikzlibrary{shadows}

\usepackage{subcaption}


\begin{document}
\title{Chess Openings Database Specification}
\author{Yoan Geran \and Colin Geniet}
\maketitle

\tableofcontents

\section{Informal specification}
The general goal of this project is to allow users to study chess openings
with the help of a large chess games database.

For that purpose, the main feature of the database is to allow searching
in the games database for games starting with a specific moves sequence
inputted by users.
More specifically, the search result will include the followings :
\begin{itemize}
\item Statistics on games matching the criteria, that is number of games and
win rates for each side (white/black).
\item Most frequent move after this initial sequence.
\item Most used openings after this initial sequence.
\item Openings matching (or less specific than) the initial sequence.
\item Some games matching those criteria (maybe sorted by ELO rating).
\end{itemize}

Searches can be further refined using additional restraints on game date, location,
event, and players name, ELO rating, nationality.

Users may also look at specific games.
When looking at a game, users may go through the game moves,
see openings matching the game and download the game as a \verb|.pgn| file.

Openings, characterized by an initial moves sequence, may be used when searching
instead of manually inputting a move sequence.
Openings may have variations, that is more specific openings.

Users may register accounts. Registered users may add new games (manually or using \verb|.pgn| files),
players, events and openings to the database. They may edit any of the above if it was
created by themselves. They may also add comments on any game, opening, \dots


\section{Conceptual Analysis}
\subsection{Entity relationship}

\tikzstyle{every entity} = [top color=white, bottom color=blue!30, 
                            draw=blue!50!black!100, drop shadow]
\tikzstyle{every weak entity} = [drop shadow={shadow xshift=.7ex, 
                                 shadow yshift=-.7ex}]
\tikzstyle{every attribute} = [top color=white, bottom color=yellow!20, 
                               draw=yellow, node distance=1cm, drop shadow]
\tikzstyle{every relationship} = [top color=white, bottom color=red!20, 
                                  draw=red!50!black!100, drop shadow]

\begin{center}
\scalebox{.80}{
\begin{tikzpicture}[every edge/.style={link}]
    \node[entity] (game) {Game};
    
    \node[relationship] (white) [below right= 0cm and 1cm of game] {white} edge node [auto] {1} (game);
    \node[relationship] (black) [above right= 0cm and 1cm of game] {black} edge node [auto,swap] {1} (game);
    \node[entity] (player) [above right= 0cm and 1cm of white] {Player} edge (white);
    \draw[link] (player) edge (black);
    
    \node[relationship] (gevent) [above left= 1cm and 0cm of game] {in event} edge node [auto] {1} (game);
    \node[entity] (event) [left= 0.5cm of gevent] {Event} edge (gevent);
    
    \node[relationship] (gopening) [left= 5cm of game] {opening} edge (game);
    \node[entity] (opening) [above= 1cm of gopening] {Opening} edge (gopening);
    \node[relationship] (variation) [above= 0.5cm of opening] {variation};
    \draw[link] (opening.70) edge (variation.283);
    \draw[link] (opening.110) edge (variation.257);
    
    \node[draw,circle,top color=white, bottom color=green!20, draw=green!50!black!100, drop shadow] (obj_is) 
         [above= 5cm of game] {D} edge (game) edge (opening) edge (event) edge (player);
    \node[entity] (object) [above= 1cm of obj_is] {Object} edge node {$\bigcup$} (obj_is);
    
    \node[relationship] (owns) [left= of object] {owns} edge [total] (object);
    \node[entity] (user) [left= of owns] {User} edge (owns);
    
    \node[relationship] (on) [above= 1cm of object] {on} edge (object);
    \node[entity] (comment) [left= of on] {Comment} edge [total] node [auto] {1} (on);
    \node[relationship] (from) [above= 0.8cm of user] {from} edge (user) edge [total] node [auto] {1} (comment);
\end{tikzpicture}
}
\end{center}

\begin{table}[ht!]
    \begin{subtable}{3.5cm}
	\begin{tabular}[t]{|cr|}
	\hline
	\multicolumn{2}{|c|}{\textbf{Object}} \\
	\hline 
	id & \em{primary key} \\
	\hline
	\end{tabular}
    \end{subtable}
    ~
    \begin{subtable}{3.5cm}
	\begin{tabular}[t]{|cr|}
	\hline
	\multicolumn{2}{|c|}{\textbf{Game}} \\
	\hline 
	moves  &                  \\
	date   & \em{optional}    \\
	site   & \em{optional}    \\
	result &                  \\
	\hline
	\end{tabular}
    \end{subtable}
    ~
    \begin{subtable}{3.5cm}
	\begin{tabular}[t]{|cr|}
	\hline
	\multicolumn{2}{|c|}{\textbf{Player}} \\
	\hline 
	first name   &                  \\
	last name    &                  \\
	Elo rating   & \em{optional}    \\
	nationality  & \em{optional}    \\
	\hline
	\end{tabular}
    \end{subtable}
    ~
    \begin{subtable}{3.5cm}
	\begin{tabular}[t]{|cr|}
	\hline
	\multicolumn{2}{|c|}{\textbf{Opening}} \\
	\hline
	moves  &                  \\
	name   &                  \\
	\hline
	\end{tabular}
    \end{subtable}
    ~
    \begin{subtable}{3.5cm}
	\begin{tabular}[t]{|cr|}
	\hline
	\multicolumn{2}{|c|}{\textbf{Event}} \\
	\hline
	name   &                  \\
	start  & \em{optional}    \\
	end    & \em{optional}    \\
	\hline
	\end{tabular}
    \end{subtable}
    ~
    \begin{subtable}{3.5cm}
	\begin{tabular}[t]{|cr|}
	\hline
	\multicolumn{2}{|c|}{\textbf{User}} \\
	\hline
	id       & \em{primary key} \\
	name     & \em{key}         \\
	password &                  \\
	\hline
	\end{tabular}
    \end{subtable}
    ~
    \begin{subtable}{3.5cm}
	\begin{tabular}[t]{|cr|}
	\hline
	\multicolumn{2}{|c|}{\textbf{Comment}} \\
	\hline
	id     & \em{primary key} \\
	text   &                  \\
	date   &                  \\
	\hline
	\end{tabular}
    \end{subtable}
\end{table}


\subsection{Integrity Constraint}
\paragraph{Moves}
\verb|moves| attributes in entities \verb|Game| and \verb|Opening| must represent
valid sequence of chess moves starting from the initial chess position.


\section{Documentation}
\subsection{Data representation}
\subsubsection{Chess moves}
Obviously, chess moves will be a very important --- and big --- part of this database.
Thus, finding an efficient representation for them is important.
The one we settled for is to represent any of the 64 squares by a character (ASCII).
A move is represented by two character, and a sequence of moves by a string.

While not human readable, this representation is quite compact, totally unambiguous,
with only one representation for any given move, and uses has constant representation
size for a move.

All this advantages are important when searching in the database for games starting
with a given move sequence, as the search is simply a prefix lookup.

The other representation we considered are :
\begin{itemize}
\item Algebraic notation is the most standard, but has problems with ambiguity
(two valid strings can represent the same move), making search much more complicated.
Additionally, it uses more space, and it has non-constant representation size.

\item Representing the start and end squares in plain text mostly keeps the advantages
of our representation while being human readable. However, it do uses twice more space.
Since having a human readable internal representation is not very useful,
we ruled it out.
\end{itemize}


\paragraph{Detailed representation}
Chess board squares are numbered from left to right, and from bottom to top,
starting with 0. Thus, \verb|a1| is numbered as 0, \verb|a8| as 7, \verb|b1| as 8, \dots
The square number is directly encoded in the six lower end bits of a byte, which is considered as a character.

For promotions, the two remaining higher end bits of the end square are used to represent the piece promoted to,
encoded as Queen=0, Knight=1, Rook=2, Bishop=3.
For instance, a pawn promoting on column \verb|a| will be represented as 6--7 for a queen promotion,
6--71 for a Knight promotion, 6--135 for a Rook promotion, \dots.

Note that castling is represented by the King move.

\section{Functional Analysis}



\end{document}