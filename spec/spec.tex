\documentclass{article}

\usepackage[utf8]{inputenc}
\usepackage[T1]{fontenc}

\usepackage{tikz-er2}
\usetikzlibrary{positioning}
\usetikzlibrary{shadows}


\begin{document}
\title{Chess Openings Database Specification}
\author{Yoan Geran \and Colin Geniet}
\maketitle

\begin{abstract}
This is the specification for a database of chess games, mainly aimed at
the study of openings.
\end{abstract}

\tableofcontents

\section{Informal specification}
The general goal of this project is to allow the user to study chess openings
with the help of a large chess games database.

For that purpose, the main feature of the database is to allow searching
in the games database for games starting with a specific moves sequence
inputted by the user.
More specifically, the search result will include the followings :
\begin{itemize}
\item Statistics on games matching the criteria, that is number of games and
win rates for each side (white/black).
\item Most frequent move after this initial sequence.
\end{itemize}







\newpage

The goal is to allow users to search in a database of chess games based on
various criteria, and to gather statistics about those games.

For each game in the database, the sequence of moves is recorded, as well as 
the result, players, date, location, event...

Chess players can also be registered with additional informations : name,
ELO rating, nationality...

The user can search games starting with a specific sequence of moves, and
add filters on game and players attributes : (game date and location,
players names, ELO, nationality ...).

The result of the search is :
\begin{itemize}
\item The number of games matching the criteria.
\item Win/loss/draw frequency those games.
\item Most used openings in those games.
\item Some games matching the criteria.
\item Most used move after the initial input sequence.
\end{itemize}

Examples of use case :
\begin{itemize}
\item The user inputs a sequence of moves. The search result is : how many games use this sequence,
how often each player wins when using it, and what is the most common way to
continue the game, as well as possibly openings starting with this sequence, if any.
\end{itemize}

Additionally, openings characterized by a sequence of moves can be recorded, and
used when searching, instead of manually inputting a move sequence. 
Any opening can also have variations, that continue the sequence of moves
defined by the opening. Variations can be seen as more specific openings, and
may have subvariations.


Users will be able to add new games, new players and new openings in the database. 


\tikzstyle{every entity} = [top color=white, bottom color=blue!30, 
                            draw=blue!50!black!100, drop shadow]
\tikzstyle{every weak entity} = [drop shadow={shadow xshift=.7ex, 
                                 shadow yshift=-.7ex}]
\tikzstyle{every attribute} = [top color=white, bottom color=yellow!20, 
                               draw=yellow, node distance=1cm, drop shadow]
\tikzstyle{every relationship} = [top color=white, bottom color=red!20, 
                                  draw=red!50!black!100, drop shadow]
\tikzstyle{every isa} = [top color=white, bottom color=green!20, 
                         draw=green!50!black!100, drop shadow]

\begin{center}
\begin{tikzpicture}[every edge/.style={link}]
    \node[entity] (game) {Game};
    \node[attribute] (gdate) [above= 1.5cm of game] {date} edge (game);
    \node[attribute] (gid) [above left= 1.4cm and 0cm of game] {\key{id}} edge (game);
    
    \node[relationship] (white) [below right= 0cm and 1cm of game] {white} edge node [auto] {1} (game);
    \node[relationship] (black) [above right= 0cm and 1cm of game] {black} edge node [auto,swap] {1} (game);
    \node[entity] (player) [above right= 0cm and 1cm of white] {Player} edge (white);
    \draw[link] (player) edge (black);
    
    \node[relationship] (gevent) [left= of game] {in event} edge node [auto] {1} (game);
    \node[entity] (event) [left= of gevent] {Event} edge (gevent);
    
    \node[relationship] (gopening) [below= of game] {opening} edge (game);
    \node[entity] (opening) [below= of gopening] {Opening} edge (gopening);
    \node[relationship] (variation) [left= of opening] {variation};
    \draw[link] (opening.172) edge (variation.10);
    \draw[link] (opening.-172) edge (variation.-10);
    
    
\end{tikzpicture}
\end{center}


\end{document}