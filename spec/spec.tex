\documentclass{article}

\usepackage[utf8]{inputenc}
\usepackage[T1]{fontenc}

\usepackage{tikz-er2}
\usetikzlibrary{positioning}
\usetikzlibrary{shadows}


\begin{document}
\title{Chess Openings Database Specification}
\author{Yoan Geran \and Colin Geniet}
\maketitle

\begin{abstract}
This is the specification for a database of chess games, mainly aimed at
the study of openings.
\end{abstract}

\tableofcontents

\section{Informal specification}
The general goal of this project is to allow users to study chess openings
with the help of a large chess games database.

For that purpose, the main feature of the database is to allow searching
in the games database for games starting with a specific moves sequence
inputted by users.
More specifically, the search result will include the followings :
\begin{itemize}
\item Statistics on games matching the criteria, that is number of games and
win rates for each side (white/black).
\item Most frequent move after this initial sequence.
\item Most used openings in those games.
\item Some games matching those criteria (maybe sorted by ELO rating).
\end{itemize}

Searches can be further refined using additional restraints on game date, location,
event, and players name, ELO rating, nationality.

Users may also look at specific games.
When looking at a game, users may go through the game moves,
see openings matching the game and download the game as a \verb|.pgn| file.

Openings, characterized by an initial moves sequence, may be used when searching
instead of manually inputting a move sequence.
Openings may have variations, that is more specific openings.

Users may register accounts. Registered users may add new games (manually or using \verb|.pgn| files),
players, events and openings to the database. They may edit any of the above if it was
created by themselves. They may also add comments on any game or opening.


\section{Entity relationship}

\tikzstyle{every entity} = [top color=white, bottom color=blue!30, 
                            draw=blue!50!black!100, drop shadow]
\tikzstyle{every weak entity} = [drop shadow={shadow xshift=.7ex, 
                                 shadow yshift=-.7ex}]
\tikzstyle{every attribute} = [top color=white, bottom color=yellow!20, 
                               draw=yellow, node distance=1cm, drop shadow]
\tikzstyle{every relationship} = [top color=white, bottom color=red!20, 
                                  draw=red!50!black!100, drop shadow]
\tikzstyle{every isa} = [top color=white, bottom color=green!20, 
                         draw=green!50!black!100, drop shadow]

\begin{center}
\begin{tikzpicture}[every edge/.style={link}]
    \node[entity] (game) {Game};
    \node[attribute] (gdate) [above= 1.5cm of game] {date} edge (game);
    \node[attribute] (gid) [above left= 1.4cm and 0cm of game] {\key{id}} edge (game);
    
    \node[relationship] (white) [below right= 0cm and 1cm of game] {white} edge node [auto] {1} (game);
    \node[relationship] (black) [above right= 0cm and 1cm of game] {black} edge node [auto,swap] {1} (game);
    \node[entity] (player) [above right= 0cm and 1cm of white] {Player} edge (white);
    \draw[link] (player) edge (black);
    
    \node[relationship] (gevent) [left= of game] {in event} edge node [auto] {1} (game);
    \node[entity] (event) [left= of gevent] {Event} edge (gevent);
    
    \node[relationship] (gopening) [below= of game] {opening} edge (game);
    \node[entity] (opening) [below= of gopening] {Opening} edge (gopening);
    \node[relationship] (variation) [left= of opening] {variation};
    \draw[link] (opening.172) edge (variation.10);
    \draw[link] (opening.-172) edge (variation.-10);
    
    
\end{tikzpicture}
\end{center}


\end{document}